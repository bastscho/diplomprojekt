	\section{Einführung}
	Die Ziele und Anforderungen in unserem Projekt Communicational haben sich im Laufe des Projekts sehr stark verändert.
	\\
	Zu Beginn sollte ein bestehendes Ticketsystem des Landesschulrates angepasst und optimiert werden, indem Funktionen ausgenommen und die Weboberfläche reduziert wird. Im Zuge der Projektdurchführung, im Besonderen der Evaluation des bestehenden Systems, wurde jedoch der Schluss gezogen, dass dies nicht ohne weiteres möglich ist. Durch die hohe Komplexität des Ursprungssystems sowie durch den Verzicht auf Dokumentation und objektorientierte Entwurfsmuster wurde der Aufwand für das Diplomprojekt, spät in der Projektdurchführung, als zu groß bemessen.
	
	\section{Planungsabweichungen}
	Das Projektziel wurde auf Grund des geschilderten Übermaßes an zeitlichem und personellem Aufwandes abgewandelt. Nun galt es, nicht wie ursprünglich geplant die Erweiterung bzw. Anpassung des bestehenden, auf OSTicket basierenden Systems des Landesschulrates Tirol durchzuführen. Es galt nun, einen Lösungsansatz für ein kompaktes, unkompliziertes Ticketsystem zu finden, das auf Webtechnologien setzt. IT-ManagerInnen an Tirols Schulen sollen IT-Infrastrukturprobleme einfach und schnell an die zuständigen SystemadministratorInnen bekannt geben können.
	
	\subsection{Projektpartner \& -betreuer}
	Die Abänderungen im Projekt wurden mit dem Projektpartner Herrn OStR. Prof. Mag. Helmut Hammerl und dem Projektbetreuer Herrn Stefan Stolz, Msc beschlossen. Beide waren nach Präsentation der Zwischenergebnisse (der hohe Arbeitsaufwand, der mit dem ursprünglichen Projektziel einher gegangen wäre) der Meinung, das Ziel des Projektes in eine akademischere Richtung zu lenken.
	
	\section{Zusammenarbeit}
	\subsection{Arbeitsaufteilung Projektteam}
	Die Änderungen im Projektscope zeigen eine starke Abweichung der Aufgabenbereiche der einzelnen Projektmitglieder.
	\\
	Die neu zugewiesenen, individuellen Projektinhalte: 
	\begin{itemize}
		\item \textbf{Jakob Tomasi:} Das erstellen eines Systementwurfs um eine Alternative für OSTicket zu bieten und der Entwurf eines JavaEE Prototypen.
		\item \textbf{Peter Pollheimer:} Die Evaluierung von OSTicket und weiteren Alternativen wie OSTicky und Katak. 
		\item \textbf{Elias Gabl:} Datenbankmodellierung und Systementwurf für die Alternative zu OSTicket.
	\end{itemize}
	\vspace{.6cm}
	Die ursprünglichen, individuellen Projektinhalte:
	\\
	\begin{itemize}
		\item \textbf{Jakob Tomasi:} Das Einlesen in OSTicket's Quellcode und die Implementierungen der gewünschten Änderungen im Backend
		\item \textbf{Peter Pollheimer:} Die Evaluierung von OSTicket und Planung/Design/Erstellung der neuen Weboberfläche
		\item \textbf{Elias Gabl:} Datenbankmodellierung, -Planung, und -Änderung  von OSTicket.
	\end{itemize}
