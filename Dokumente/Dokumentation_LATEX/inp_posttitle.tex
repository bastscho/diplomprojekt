%todo: Titelblatt ändern, gibt mehrere, ein richtiges (mit unseren Namen) wurde bereits erstellt.
%-------------
%todo: setzen von wichtigen Todo's priorisieren und die wichtigsten zuerst abarbeiten
\chapter*{Eidesstattliche Erklärung}
Ich erkläre an Eides statt, dass ich die vorliegende Diplomarbeit selbst verfasst und keine anderen als die angeführten Behelfe verwendet habe. Alle Stellen, die wörtlich oder inhaltlich den angegebenen Quellen entnommen wurden, sind als solche kenntlich gemacht.
Ich bin damit einverstanden, dass meine Arbeit öffentlich zugänglich gemacht wird.

\vspace{1cm}
\begin{tabular}{c c c}
	& \hspace{4cm} & \\\cline{1-1}
	Ort, Datum & & \\
	\vspace{2cm}
	& & \\\cline{1-1}
	Jakob Tomasi
	\vspace{2cm}
	& & \\\cline{1-1}
	Peter Pollheimer
	\vspace{2cm}
	& & \\\cline{1-1}
	Elias Gabl 
\end{tabular}

\chapter*{Abnahmeerklärung}
Hiermit bestätigt der Auftraggeber, dass das übergebene Produkt dieser Diplomarbeit den dokumentierten Vorgaben entspricht. Des Weiteren verzichtet der Auftraggeber auf unentgeltliche Wartung und Weiterentwicklung des Produktes durch die Projektmitglieder bzw. die Schule.

\vspace{1cm}
\begin{tabular}{c}
	\\\cline{1-1}
	Ort, Datum\\
	\vspace{2cm}
	\\\cline{1-1}
	Auftraggeber
\end{tabular}	


\chapter*{Vorwort}
Beauftragt wurde das Projektteam von \getHammerl\ im Namen des Landesschulrates Tirol. Der Kontakt mit \getHammerl\ wurde von \getAlex\ hergestellt. In einer ersten Besprechung erläuterte \getHammerl\ die Problemstellung, ein Ticketsystem verwenden zu müssen, welches auf Mobilgeräten kaum effektiv eingesetzt werden kann. Das Projektteam und \getHammerl\ einigten sich darauf, im Rahmen des Diplom- und Abschlussprojektes des Teams dessen Dienste in Anspruch zu nehmen und im Gegenzug die Projektbetreuung zu übernehmen.



\chapter*{Kurzfassung}
Ein bereits bestehendes Ticketsystem des Landesschulrats Tirol soll durch Änderungen funktional vereinfacht und optisch aufbereitet werden. Mit diesem System können IT-ManagerInnen Probleme melden, die die IT Infrastruktur einer Schule betreffen. Nach den ersten abgeschlossenen Phasen des Projektes, wurde jedoch vom Projektteam festgestellt, dass der Umfang des vorgegebenen Ticketsystems zu umfangreich ist.\newline
Um die Projektziele zu erreichen und Änderungen zu bewerkstelligen, liegt der Fokus auf einer vom Projektteam verfassten Dokumentation um mit dem vorgegebenen Quellcode bestmöglichst zu arbeiten.

\chapter*{Abstract}
As of today, the education authority of Tyrol uses the open source ticketing system OSTicket. It enables IT-Managers to report problems with school’s IT infrastructure. The scope of this project is to simplify the usage of OSTicket by creating a new web-based interface and removing unused functions. However, while getting used to the internal works of OSTicket, the team realized it was way too crufty.\newline
In order to meet the project’s requirements, it became necessary to find an alternative system, which enables the team to deliver a ticketing system capable of adaption in an efficient manner.


\chapter*{Zusammenfassung}
Communicational erweitert ein bestehendes System des Landesschulrats Tirol, basierend auf \getOst\. \getOst\ ist ein Webbasiertes Open-Source Ticketsystem. Es wird verwendet, um IT-ManagerInnen an Tirols Schulen Probleme ihrer IT-Infrastruktur an die Zuständigen SystemadministratorInnen bekannt zu geben. Dies erfolgt unter der Domain itsys-tirol.at. Dieses Portal unterstützt nur Geräte mit großen Bildschirmen (ab 1024 Pixel Bildschirmbreite). Dies erschwert die Verwendung mit mobilen Geräten erheblich. Durch die im Rahmen dieses Projektes vorgenommenen Anpassungen an der Benutzeroberfläche wird die Benutzung mit verschiedensten Endgeräten wie z.B. Smartphones und Tablets ermöglicht.\newline
Des Weiteren befindet der Projektpartner - OStR. Prof. Mag. Hammerl Helmut - die Grundkonfiguration von \getOst\ für zu voluminös. Deshalb sollen möglichst viele nicht verwendete Funktionen im angepassten User-Interface weggelassen werden, um die Bedienbarkeit und Userfreundlichkeit zu erhöhen.\newline
Nach vielen Stunden der Einarbeitung in \getOst\ wurde dem Projektteam klar, dass dieses durch seine allgemeine Beschaffenheit wie die fehlende Dokumentation des Codes und die Vermischung von Programmlogik und HTML-Elementen schlecht für die vom Projektteam geplante Anpassungen geeignet ist. Aufgrund dessen wurde eine Alternative gesucht, mit der die im Rahmen des Projektes durchzuführenden Änderungen effizienter zu bewerkstelligen sind. Gefunden wurde ein Fork von \getOst\ mit dem Namen Katak (eine Abspaltung, die von einem anderen Entwicklerteam durchgeführt und gewartet wird), welcher für die Projektbedürfnisse eine hervorragende Alternative darstellt.


