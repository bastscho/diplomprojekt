
\chapter{Alternative: PHP Applikation}
\renewcommand{\currentAuthor}{Peter Pollheimer}
\label{php}
Die Alternative, mit PHP von Grund auf ein System zu entwickeln und die wichtigsten Anforderungen wie Usability, Mobility und Adaptability zu berücksichtigen, stellt auch eine Möglichkeit dar. Es folgen einige Vorteile, die dieser Lösungsansatz bieten könnte:

\begin{itemize}
	\item Der Unterricht im IT-Kolleg spezialisiert sich vor allem im dritten Semester auf PHP und Entwurfsmuster.
	\item Die bereits vorhanden Prototypen können verwendet und angepasst werden.
	\item Die erstellten Use-Cases in dieser Evaluation schaffen bereits Klarheit über die Anwender und deren Anforderungen an das System.
	\item Wartungsprogrammierer für dieses System zu finden ist leichter, da PHP Programmierer stärker vertreten sind als Java EE7 Entwickler.
	\item Das Arbeiten mit einer MySQL Datenbank wird im dritten und vierten Semester im IT-Kolleg intensiv thematisiert.
	\item Die Expertise der Lehrpersonen im Bereich Programmierung und Datenbankmodellierung stehen  während der gesamten Zeit zur Verfügung.
\end{itemize}
Die wichtigsten Punkte, die beachtet werden sollten um ein System mit PHP zu erstellen, damit alle Anforderungen erfüllt werden, können folgendermaßen zusammengefasst werden:
\begin{itemize}
	\item Das verwenden eines geeigneten Entwurfsmusters wie zum Beispiel MVC.
	\item Das Erstellen einer Dokumentation damit die Erweiterbarkeit des Systems gegeben ist. 
	\item Striktes Einhalten von Sprachkonventionen für das gesamte System.
\end{itemize}