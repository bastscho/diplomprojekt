\chapter{Alternative PHP Applikation}
\label{php}
\def \currentAuthor{Peter Pollheimer}
Die Alternative mit PHP ein System zu entwickeln und die wichtigsten Funktionen wie Usability, Mobility und Adaptability zu berücksichtigen, bietet viele Vorteile wie:

\begin{itemize}
	\item Der Unterricht im IT-Kolleg spezialisiert sich vor allem im dritten Semester auf PHP und Entwurfsmuster.
	\item Die bereits vorhanden Prototypen können verwendet und angepasst werden.
	\item Die erstellten Use-Cases in dieser Evaluation schaffen bereits Klarheit über die Anwender und deren Anforderungen an das System.
	\item Wartungsprogrammierer für dieses System zu finden ist leichter, da PHP Programmierer stärker vertreten sind als Java EE7 Entwickler.
	\item Das arbeiten mit einer MySQL Datenbank wird im dritten und vierten Semester im IT-Kolleg intensiv thematisiert und besprochen.
	\item Die Expertise der Lehrpersonen im Bereich Programmierung und Datenbankmodellierung stehen  während der gesamten Zeit zur Verfügung.
\end{itemize}
Die wichtigsten Punkte die beachtet werden sollten um ein System mit PHP zu erstellen damit alle Anforderungen erfüllt werden, können folgendermaßen zusammengefasst werden:
\begin{itemize}
	\item Das verwenden eines geeigneten Entwurfsmusters zum Beispiel MVC.
	\item Das Erstellen einer Dokumentation damit die Erweiterbarkeit des Systems gegeben ist. 
\end{itemize}
