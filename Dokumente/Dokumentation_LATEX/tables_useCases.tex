\def \currentAuthor{Das Projektteam}
\renewcommand{\arraystretch}{1.3}

%todo: Gender prüfung 
\begin{table}[h]
	\begin{tabular}{|p{3cm}|p{10.7cm}|}
		\hline
		Name & Anmelden am Portal\\
		\hline
		ID &  C00\\
		\hline
		Beschreibung & Ein/Eine Anwender\_in oder Systembetreuer\_innen nutzt seine/ihre Login-Daten um sich in das System einzuwählen.\\
		\hline
		Akteure & Anwender\_innen, Systembetreuer\_innen\\
		\hline
		Häufigkeit & 5/5\\
		\hline
		Auslöser & Der/Die Anwender\_innen möchte das Portal benutzen.\\
		\hline
		Bedingungen & Anmeldedaten vorhanden und Rechte vergeben. \\
		\hline
		Endzustand & Eine Session wurde geöffnet und der/die Anwender\_innen kann das System nutzen.\\
		\hline
		Hauptablauf & 1. Benutzerdaten eingeben \newline2. Login Vorgang  initiieren \\
		\hline
		Ausnahmen & Error -> Anmeldedaten prüfen. Error -> Anmeldedaten neu beziehen.\\
		\hline
	\end{tabular}
	\caption{tab:Use-Case C00}
\end{table}
\label{tab:tab:Use-Case C00}
\vspace{-.5cm}
\paragraph{C00:}
Der Benutzer wird aufgefordert seine Benutzerdaten einzugeben.
Hat er die Daten richtig eingegeben wird er angemeldet. Sind die Daten falsch kommt er zurück zur Dateneingabe.


\begin{table}[h]
	\begin{tabular}{|p{3cm}|p{10.7cm}|}
		\hline
		Name & Erstellung eines Standarttickets\\
		\hline
		ID & C01\\
		\hline
		Beschreibung & Ein/Eine Anwender\_in meldet sich am Portal an und möchte ein neues Netzwerkgerät bestellen. Dieser/Diese eröffnet ein neues Standartticket und beschreibt den Grund für die Anschaffung. Der/Die Anwender\_in reicht das Ticket ein.\\
		\hline
		Akteure & Anwender\_in\\
		\hline
		Häufigkeit & 2/5\\
		\hline
		Auslöser & Es wird eine Komponente/Ressource für die Schule benötigt.\\
		\hline
		Bedingungen & C00\\
		\hline
		Endzustand & Ticket wurde in der Datenbank gespeichert.\\
		\hline
		Hauptablauf & 1. Formular öffnen \newline2. Formular ausfüllen \newline3. Ticket überprüfen \newline4. Ticket einreichen\\
		\hline
		Ausnahmen & Error -> Ticket kann nicht abgegeben werden -> C01 erneut ausführen\\
		\hline
	\end{tabular}
	\caption{tab:Use-Case C01}
\end{table}
\label{tab:tab:Use-Case C01}

\paragraph{C01:}
Der Benutzer muss zuerst ein Formular öffnen, um ein Ticket erstellen zu können. Anschließend muss das Formular ausgefüllt werden. 
\\
Nach dem Überprüfen des Tickets kann der Benutzer sich entscheiden ob er das Ticket abschickt oder ob er Änderungen vornehmen möchte. 
\\
Wenn das Abschicken fehlgeschlagen ist kommt er zum Anfang zurück. Wenn das abschicken erfolgreich war, wird das Ticket in der Datenbank gespeichert.
\newpage
\begin{table}[h]
	\begin{tabular}{|p{3cm}|p{10.7cm}|}
		\hline
		Name & Erstellung eines Incident\\
		\hline
		ID &  C02\\
		\hline
		Beschreibung & Ein/Eine Anwender\_in meldet sich am Portal an und öffnet ein Ticket, er/sie meldet ein Problem, das er/sie selbst nicht lösen kann. Der/Die Anwender\_in markiert das Ticket als Incident und speichert es.\\
		\hline
		Akteure & Anwender\_in\\
		\hline
		Häufigkeit & 4/5\\
		\hline
		Auslöser & Ein für den/die Anwender\_in nicht lösbares technisches Problem.\\
		\hline
		Bedingungen & C00, Auslösen\\
		\hline
		Endzustand & Incident wurde in der Datenbank gespeichert und der/die Systembetreuer\_in hat eine Benachrichtigung erhalten.\\
		\hline
		Hauptablauf & 1. Formular öffnen \newline2. Formular ausfüllen \newline3. Incident überprüfen \newline4. Incident einreichen\\
		\hline
		Ausnahmen & Error -> Ticket kann nicht abgegeben werden -> C02 erneut ausführen.\\
		\hline
	\end{tabular}
	\caption{tab:Use-Case C02}
\end{table}
\label{tab:tab:Use-Case C02}

\paragraph{C02:}
Der Benutzer muss zuerst ein Formular öffnen um einen Incident erstellen zu können. Anschließend muss das Formular ausgefüllt werden. Nach dem Überprüfen des Incident kann der Benutzer sich entscheiden, ob er den Incident abschickt oder ob er Änderungen vornehmen möchte.
\\
Wenn das Abschicken fehlgeschlagen ist kommt er zum Anfang zurück. Wenn das abschicken erfolgreich war wird der Incident in der Datenbank gespeichert und der Systembetreuer erhält eine Benachrichtigung.

\newpage
\begin{table}[h]
	\begin{tabular}{|p{3cm}|p{10.7cm}|}
		\hline
		Name & Ticketstatus prüfen\\
		\hline
		ID & C03\\
		\hline
		Beschreibung & Ein/Eine Anwender\_in meldet sich am Portal an und sieht sich seine/ihre eigenen Tickets an. Er/Sie kann einsehen ob der/die Systembetreuer\_in das Ticket erhalten hat und wie weit die Bestellung fortgeschritten ist.\\
		Akteure & Anwender\_in\\
		\hline
		Häufigkeit & 3/5\\
		\hline
		Auslöser & Anwender\_in hat eine Anfrage erhalten.\\
		\hline
		Bedingungen & C00, zu überprüfendes Ticket muss vorhanden sein.\\
		\hline
		Endzustand & Keine Veränderung am System\\
		\hline
		Hauptablauf & 1. Ticket öffnen \newline2. Status ansehen\\
		\hline
		Ausnahmen & Error -> Ticket kann nicht überprüft werden -> C03 erneut ausführen\\
		\hline
	\end{tabular}
	\caption{tab:Use-Case C03}
\end{table}
\label{tab:tab:Use-Case C03}

\paragraph{C03:}
Der Benutzer muss das Formular öffnen um den Status zu sehen. Ist das Öffnen fehlgeschlagen kommt er wieder zum Ausgspunkt und kann es nochmal versuchen.

\newpage
\begin{table}[h]
	\begin{tabular}{|p{3cm}|p{10.7cm}|}
		\hline
		Name & Ticketlöschung beantragen\\
		\hline
		ID & C04\\
		\hline
		Beschreibung & Ein/Eine Anwender\_in meldet sich am Portal an und beantragt die Löschung seines Tickets.\\
		\hline
		Akteure & Anwender\_in\\
		\hline
		Häufigkeit & 1/5\\
		\hline
		Auslöser & Ticket hat sich erübrigt, Ticket ist falsch\\
		\hline
		Bedingungen & C00, zu löschendes Ticket muss vorhanden sein\\
		\hline
		Endzustand & Systembetreuer\_in hat die Anfrage auf Löschung erhalten\\
		\hline
		Hauptablauf & 1. Ticket öffnen \newline2. Löschung beantragen\\
		\hline
		Ausnahmen & Error -> Löschvorgang nicht erfolgreich -> C04 erneut ausführen\\
		\hline
	\end{tabular}
	\caption{tab:Use-Case C04}
\end{table}
\label{tab:tab:Use-Case C04}

\paragraph{C04:}
Um die Löschung beantragen zu können muss der Benutzer zuerst das Ticket öffnen. Danach kann er die Löschung beantragen. Schlägt dies fehl kommt er wieder zurück an den Anfang und kann es nochmal versuchen. War der Antrag auf Löschung erfolgreich erhält der Systembetreuer eine Anfrage zur Löschung.

\newpage
\begin{table}[h]
	\begin{tabular}{|p{3cm}|p{10.7cm}|}
		\hline
		Name & Ticket löschen\\
		\hline
		ID & C05\\
		\hline
		Beschreibung & Der/Die Systembetreuer\_in meldet sich am Portal an und kümmert sich um Löschanfragen.\\
		\hline
		Akteure & Systembetreuer\_in\\
		\hline
		Häufigkeit & 1/5\\
		\hline
		Auslöser & C04\\
		\hline
		Bedingungen & C00, zu löschendes Ticket muss vorhanden sein.\\
		\hline
		Endzustand & Das Ticket wurde entfernt.\\
		\hline
		Hauptablauf & 1. Ticket öffnen \newline2. Ticket löschen\\
		\hline
		Ausnahmen & Error -> Löschvorgang nicht erfolgreich -> C05 erneut ausführen.\\
		\hline
	\end{tabular}
	\caption{tab:Use-Case C05}
\end{table}
\label{tab:tab:Use-Case C05}

\paragraph{C05:}
Um ein Ticket zu löschen muss der Systembetreuer das Ticket öffnen und löschen. Schlug dies fehl kommt er wieder zurück an den Anfang und kann es nochmal versuchen. War das Löschen erfolgreich wurde das Ticket aus der Datenbank entfernt.

\newpage
\begin{table}[h]
	\begin{tabular}{|p{3cm}|p{10.7cm}|}
		\hline
		Name & Ticketstatus ändern\\
		\hline
		ID &  C06\\
		\hline
		Beschreibung & Der/Die Systembetreuer meldet sich am Portal an und sieht die Tickets der Anwender\_innen in seinem/ihrem Cluster\\
		\hline
		Akteure & Systembetreuer\_in\\
		\hline
		Häufigkeit & 4/5\\
		\hline
		Auslöser & C01\\
		\hline
		Bedingungen & C00, das Ticket muss vorhanden sein.\\
		\hline
		Endzustand & Der Ticketstatus wurde geändert.\\
		\hline
		Hauptablauf & 1. Ticket öffnen \newline2. Ticketstatus ändern\\
		\hline
		Ausnahmen & Error -> Statusänderung nicht erfolgreich -> C06 erneut ausführen.\\
		\hline
	\end{tabular}
	\caption{tab:Use-Case C06}
\end{table}
\label{tab:tab:Use-Case C06}

\paragraph{C06:}
Um den Ticketstatus ändern zu können muss der Systembetreuer das Ticket öffnen und ändern. Schlug dies fehl kommt er wieder zurück an den Anfang und kann es nochmal versuchen. War das ändern erfolgreich wurde der Ticketstatus geändert.

\newpage
\begin{table}[h]
	\begin{tabular}{|p{3cm}|p{10.7cm}|}
		\hline
		Name & Incident bearbeiten\\
		\hline
		ID &  C07\\
		\hline
		Beschreibung & Der/Die Systembetreuer\_in erhält eine E-Mail-Benachrichtigung über einen Incident. Er/Sie klickt aud den mitgelieferten Link und meldet sich am Portal an. Er/Sie bearbeitet den Incident mit erhöhter Priorität.\\
		\hline
		Akteure & Systembetreuer\_in\\
		\hline
		Häufigkeit & 3/5\\
		\hline
		Auslöser & C02\\
		\hline
		Bedingungen & C00, der Incident muss vorhanden sein.\\
		\hline
		Endzustand & Der Incident ist abgearbeitet.\\
		\hline
		Hauptablauf & 1. Benachrichtigung erhalten \newline2. Incident öffnen \newline3. Incident bearbeiten \newline4. C08\\
		\hline
		Ausnahmen & Error -> Incident öffnen nicht erfolgreich -> C07 erneut ausführen.\\
		\hline
	\end{tabular}
	\caption{tab:Use-Case C07}
\end{table}
\label{tab:tab:Use-Case C07}

\paragraph{C07:}
Um ein Ticket schließen zu können muss der Systembetreuer das Ticket öffnen um es dann zu schließen. Schlug dies fehl kommt er wieder zurück an den Anfang und kann es nochmal versuchen. War das schließen erfolgreich ist das Ticket geschlossen.

\newpage
\begin{table}[h]
	\begin{tabular}{|p{3cm}|p{10.7cm}|}
		\hline
		Name & Ticket schließen\\
		\hline
		ID &  C08\\
		\hline
		Beschreibung & Der/Die Systembetreuer\_in meldet sich am Portal an und schließt ein Ticket bzw. einen Incident nach dessen Erledigung.\\
		\hline
		Akteure & Systembetreuer\_in\\
		\hline
		Häufigkeit & 5/5\\
		\hline
		Auslöser & C01, C02\\
		\hline
		Bedingungen & C00, das Ticket muss vorhanden und abgearbeitet sein.\\
		\hline
		Endzustand & Ticket ist geschlossen\\
		\hline
		Hauptablauf & 1. Ticket öffnen \newline2. Ticket schließen\\
		\hline
		Ausnahmen & Error -> Ticket konnte nicht geschlossen werden -> C08 erneut ausführen.\\
		\hline
	\end{tabular}
	\caption{tab:Use-Case C08}
\end{table}
\label{tab:tab:Use-Case C08}

\paragraph{C08:}
Hat der User das Ticket erfolgreich geöffnet und bearbeitet, so muss das Ticket wieder geschlossen werden. Nach erfolgreichem Schließen des Tickets, kommt der Anwender/die Anwenderin wieder zurück zum Ausgangspunkt.


\renewcommand{\arraystretch}{1}